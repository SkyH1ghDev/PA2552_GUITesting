\documentclass{article}
\usepackage{hyperref}
\usepackage{array}
\usepackage{booktabs, caption}
\usepackage[flushleft]{threeparttable}
\usepackage{float}
\usepackage[top=2.5cm, bottom=2.5cm, left=2.5cm, right=2.5cm]{geometry}

\bibliographystyle{IEEEtran}

\title{PA2552 - Software Testing \\
	\large GUI Testing 
	}

\author{Christoffer Bohman \\
	MScEng: Game Engineering \\
	Blekinge Institute of Technology \\
	\\
	Contact: \\
	chbh22@student.bth.se / student@skyh1gh.dev \\
	\\
	Revisions and git history: \\
	\href{https://github.com/SkyH1ghDev/PA2552_GUITesting}{GitHub} \\
	}

\date{\today}

\begin{document}

\maketitle

\newpage

\section{Test Cases \& Acceptance Criteria}

\begin{itemize}
	\item As a user I want to be able to access the Godot webpage.
	\subitem{\textbf{Acceptance:}} The website is accessible through a standard web browser. \\

	\item As a User I want to be able to download the latest version of Godot to my machine
	\subitem{\textbf{Acceptance:}} The software has been successfully downloaded to the local machine. \\

	\item As a contributor I want to be able to know how to contribute to the project.
	\subitem{\textbf{Acceptance:}} The relevant learning resources are available and accessible to the contributor. \\

	\item As a corporate sponsor I want to have a way for users to access my tools from the Godot website.
	\subitem{\textbf{Acceptance:}} The information of the sponsors as well as their related links can be found and accessed through the website. \\ 

	\item As a user I want to be able to access the Godot Documentation
	\subitem{\textbf{Acceptance:}} The website is accessible through a standard web browser. \\

	\item As the Godot foundation I want to expose a way for users to show support through donations
	\subitem{\textbf{Acceptance:}} A donation webpage is available and a donation can be sent through it. \\

	\item As a user I want to be able to search the documentation according to my needs
	\subitem{\textbf{Acceptance:}} A search box should be usable and the results should be relevant to the search keyword. \\

	\item As a contributor I want to be able to report bugs
	\subitem{\textbf{Acceptance:}} The Godot webpage has a feature that allows users to report defects in the Godot project. \\

	\item As a user I want to be able to rollback or download older versions of the engine.
	\subitem{\textbf{Acceptance:}} A webpage with the older versions should be accessible and the links should be working. \\

	\item As a user I want to be able to use a translated version of the webpage
	\subitem{\textbf{Acceptance:}} An option to change the language of the webpage should be accessible. The most common languages should also be available. \\
\end{itemize}

\begin{table}[H]
	\caption{Abstract and Concrete Lean Testing Principles}
	\begin{center}
		\begin{tabular}{| m{24em} | m{6em} | m{6em} |} 
			\hline
			Test Case & Development Time in m (approx) & Execution Time in s (approx) \\
			\hline
			\hline
			Access Godot Webpage & 10 & 5 \\
			Download Latest Version & 5 & 7 \\
			Access Contribution Page & 10 & 10 \\
			Verify Sponsors Pages Work & 25 & 26 \\
			Access Engine Documentation & 10 & 9 \\
			Access Donation Page & 20 & 17 \\
			Search For Specific Item In Documentation & 15 & 17 \\
			Find Resources For Bug Reporting & 25 & 20 \\
			Finding Older Versions of Godot & 20 & 20 \\
			Changing To Translated Version & 10 & 10 \\
			\hline
		\end{tabular}
	\end{center}
	\begin{tablenotes}
		\small
		\item Table 1: Values from the GUI testing suite
	\end{tablenotes}
\end{table}

\newpage

\section{Discussion}

My test suite is based on testing the functionality and usage of the Godot foundations webpage for the Godot Game Engine.
The reasoning behind this decision was partially due to the fact that I am studying a game engine focused programme, and
partially because it is easy to reason about the different requirements of an open source development project.

In the list below, I will describe and explain why I chose these specific features for my suite, why they are relevant
to test which resultingly also partly implies why they are of high quality. 

\begin{itemize}
	\item Access Godot Webpage
	\subitem{-} Accessing the main component of any application is a requirement before being able to test anything else.
	That same principle applies here and this theme can be tracked throughout my testing as this specific test case serves
	as the isolating member that restores the original state of the Godot webpage after every test. This case is fairly simple
	all in all making it easy to read and by implication also rewrite should the agile environment require it. This scenario
	only encompasses the opening of the firefox web browser, search of the godot website and handling the closure of the web browser.

	\item Download Latest Version
	\subitem{-} As with any website that hosts any kind of file, a way to download or retrieve that file is also generally a 
	requirement. This test case handles the scenario where the user wants to download the latest version of the Godot Engine.
	A limitation with my test implementation is that the file does not get downloaded as it should be per the defined requirements
	above, this is to avoid any rate limitation and unecessary retrieval of the same large file over several test suite runs.

	\item Access Contribution Page
	\subitem{-} Free Open Source Software is heavily reliant on active maintainers and contributors in order to stay alive and up 
	to date. Resources for anyone that wants to take part in this should be available in order to create an environment where new
	contributors may join in a systematic and productive way. This case covers the availability of a contribution page as well as 
	makes sure the documentation may be accessed through the main webpage.

	\item Verify Sponsors Pages Work
	\subitem{-} Third-party financial aid, especially for non-profit foundations, can boost the possibilities of reaching certain
	goals that are being aimed for. Consequently, a means to garner sponsorships and financial support is through advertisement of
	each sponsors personal page. In this test, the ability to see and access the different webpages is being tested. For a more accurate
	result, a variety sponsor links are being tested to ensure that no link is invalid or doesn't work.

	\item Access Engine Documentation
	\subitem{-} For this software to be usable, the Godot foundation needs to guarantee access to materials that enable developers to 
	work with the engine. The responsibility of this test is to ascertain the possibility of finding and reading such materials.
	
	\item Access Donation Page
	\subitem{-} 

	\item Search For Specific Item In Documentation
	\subitem{-} 

	\item Find Resources For Bug Reporting
	\subitem{-} 

	\item Finding Older Versions Of Godot
	\subitem{-} 

	\item Changing To Translated Version
	\subitem{-} 

\end{itemize}

\end{document}